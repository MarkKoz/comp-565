% !TEX output_directory = ./.temp/report
% !TEX options = --shell-escape

\documentclass[a4paper, 12pt]{scrartcl}
\usepackage{lmodern}
\usepackage[T1]{fontenc}
\usepackage[utf8]{inputenc}

\usepackage{hyperref}
\hypersetup{bookmarksnumbered}
\usepackage{bookmark}

% Remove space reserved for unused fields.
% https://tex.stackexchange.com/a/134857
\makeatletter
\renewcommand{\@maketitle}{\null\vskip 2em
\begin{center}
  \ifx\@subject\@empty \else
    {\subject@font \@subject \par}
    \vskip 1.5em
  \fi
  \titlefont\huge \@title\par
  \vskip .5em
  {\ifx\@subtitle\@empty\else\usekomafont{subtitle}\@subtitle\par\fi}%
\end{center}
\vskip 2em}
\makeatother

\title{OpenGL Basics}
\subject{COMP 565 - Advanced Computer Graphics}
\subtitle{Assignment 1}
\date{}
\author{}

\begin{document}

\maketitle

\section{Objectives}
The objective is to render a triangle which demonstrates fragment interpolation. This means a fragment's colour is a combination of the three vertices' colours. Its position relative to a vertex determines how much that of vertex's colour is present in the fragment's colour.

Initially, the first vertex is red, the second green, and the third blue. The vertex colours need to be updated every frame so that they smoothly transition to black and then back to their original colours. Because the vertex colours change, the results of fragment interpolation should also be affected.

\section{OpenGL}
\subsection{Graphics Pipeline}
The graphics pipeline in OpenGL ultimately transforms 3D coordinates into coloured 2D pixels. The pipeline can be split into two main sections: transforming 3D coordinates to 2D coordinates, and transforming 2D coordinates to coloured pixels. The pipeline has six stages which are described below.

\paragraph{Vertex Shader}
The vertex shader is the first stop in the graphics pipeline. The shader receives a single vertex as an input and generates a single vertex as an output. It may transform 3D coordinates into other 3D coordinates if desired. It can also modify other arbitrary attributes of the vertex (such as colour).

\paragraph{Primitive Assembly}
OpenGL has \textit{primitives}, which are different ways to interpret a set of vertices. For example, the vertices can be interpreted as a triangle or a line. The primitive assembly stage converts the vertices into primitives. Essentially it forms some shape from the vertices.

\paragraph{Geometry Shader}
Once a primitive is formed, the geometry shader can be used to form new primitives based on the given vertices for a primitive formed by primitive assembly.

\paragraph{Rasterisation}
The rasterisation stage converts primitives into pixels that will be rendered on the screen. This results in \textit{fragments}, which are objects representing pixel data for OpenGL. This stage also performs clipping to remove fragments that will not be visible on screen.

\paragraph{Fragment Shader}
The fragment shader takes a single fragment as an output and produces a single fragment as an output. The shader can modify the fragment's colour.

\paragraph{Tests and Blending}
Though the fragment shader sets the colour of a fragment, the final stage can still affect the final colour of fragments. This stage blends fragments together depending on their alpha values, which determine their opacity. It checks the depth of fragments to determine their positions relative to other fragments (behind or in front), discarding fragments if needed.

\subsection{Vertex Inputs}
Vertex coordinates are expected to be 3D. OpenGL processes \textit{normalised device coordinates} (NDC), which are coordinates in the range $[-1,1]$. In this coordinate system, the origin is $(0,0,0)$. Positive $y$ points up, positive $x$ points right, and $z$ determines the depth.

\textit{Vertex buffer objects} (VBO) are used to stored vertices in the GPU's memory. There are different ways in which the GPU can manage this data. For this assignment's objective, the vertex data needs to be updated every frame. Hence, \texttt{GL\_STREAM\_DRAW} is the ideal option for managing the vertex data. The can be created with \texttt{glGenBuffers}, bound with \texttt{glBindBuffer} (so that the \texttt{GL\_ARRAY\_BUFFER} target refers to this buffer), and populated with \texttt{glBufferData}.

\subsubsection{Attributes}
The vertex data can be in any arbitrary format, as long as the shader will be told how to interpret the data. For this assignment, each vertex consists of six values: the first three are the position coordinates, and the last 3 are the RGB values for that vertex's colour. The position and the colour are known as \textit{vertex attributes}. In this case, both are 3D vectors, or \texttt{vec3}s.

The way in which attributes are interpreted is defined using the \texttt{glVertexAttribPointer} function. It specifies the attribute's size, its values' data type, its starting offset in the vertex buffer, and its stride (space between consecutive attributes).

\subsection{Shaders}
Shaders are small programs that run on the GPU and are used to process data within the graphics pipeline. They are written using GLSL, the OpenGL Shading Language, which resembles C. The overview of the graphics pipeline mentioned three shaders: vertex shaders, geometry shaders, and fragment shaders. OpenGL requires at least a vertex shader and a fragment shader; there are no defaults for these two.

As mentioned earlier, shaders have inputs and outputs. These need to be defined by the shaders. An output from one shader can be used as an input for a shader in a subsequent stage of the pipeline.

\subsubsection{Vertex Shader}
The vertex shader takes a single vertex as an input and outputs a single vertex. For this assignment, the vertices have two attributes: position and colour. The shader sets the output position by assigning it to the predefined \texttt{gl\_Position} variable. However, this expects a \texttt{vec4}. Hence, the input position must be converted from \texttt{vec3}. To do this, the three input components are copied, and the $w$ component, which is used for perspective division, is set to $1.0$.

As for the colour, the shader simply outputs the colour as given. This will later be used by the fragment shader.

\subsubsection{Fragment Shader}
The fragment shader controls the colour of the pixels. For this assignment, the shader takes as an input the colour that was output by the vertex shader. The fragment shader does modify the colour, but it does need to convert it to a \texttt{vec4}, since that is the expected output format. The fourth component is the alpha channel, which determined the opacity. In this case, all fragments are fully opaque, so the alpha is set to $1.0$.

\subsection{Shader Program}
Shaders first have to be compiled. Once compiled, shaders have to be linked so that the outputs of one shader can be used as inputs for the other. The result of linkage is the \textit{shader program}. The \texttt{glUseProgram} function has to be called in order to use the resulting program, which in turn tells OpenGL which shaders it should be using.

\subsection{Vertex Array Object}
A \textit{vertex array object} (VAO) is useful for saving configurations for vertex attributes and buffers. When a VAO is bound, the aforementioned configurations will be tracked by the VAO so that the VAO can be re-used at a later point and restore those configurations for OpenGL. It becomes especially useful when needing to use different sets of configurations. Similar to a shader program, binding the VAO (via \texttt{glBindVertexArray}) will tell OpenGL which configuration to use while rendering.

\subsection{Drawing}
Once a program is active and a VAO is bound, OpenGL is ready to draw the vertices. This can be done with the \texttt{glDrawArrays} function. It has to be given the type of primitive to draw. In this case, a triangle, so \texttt{GL\_TRIANGLES}. It also needs to know the amount of vertices (3) and the starting index of the vertex array (0).

\section{Conclusions}
\subsection{Challenges}
The main challenge was getting the vertex colours to update every frame. The assigned reading didn't really explain how to do this practically. The uniforms approach shown does not scale up because it cannot specify a different value depending on the input vertex. Thus, I had to research on my own how to update vertices. I naïvely attempted to write to the original vertices array, but quickly discovered that does not work. Therefore, I figured I had to update the vertex buffer object instead. I successfully achieved this with the \texttt{glBufferSubData} function, but ultimately settled on an alternative approach, the \texttt{glMapBufferRange} function.

\subsection{Learning Outcomes}
I learned about the OpenGL graphics pipeline; I went from zero knowledge to having a high-level overview of OpenGL. I learned the basics of shaders: GLSL, inputs, outputs, uniforms, compilation, linking, and shader programs. I also learned about vertices: vertex attributes, vertex buffer objects, vertex array objects, and element buffers objects. Finally, I learned how to update buffers using the two functions described in the previous section.

\end{document}

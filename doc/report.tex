\documentclass[a4paper, 12pt]{scrartcl}
\usepackage[assign-num=3]{report}

\title{Comparison of Game Engines}

\begin{document}

\maketitle

\section{Introduction}
A game engine is a tool for developers of video games. It is software framework which abstracts away core parts of a video game such as rendering, physics, input handling, sound, AI, etc. It typically includes software libraries and other software in the form of a software development kit (SDK), which contains additional tools revolving around the engine's ecosystem (level editor, asset packager, and so on).

A game engine enables developers to create games more efficiently, since a lot of the fundamental components of a game are available for use out of the box. This saves developers the time of having to implement all these core features from scratch, which could otherwise add a tremendous amount of complexity to the project. For example, game engines may offer cross-platform capabilities in a more or less transparent way to the developers. This removes the necessity to deal with the plethora of different platform-specific APIs, which allows developers and publishers to reach a broader market more easily.

\subsection{History}
Game engines were not immediately prevalent in the industry when it first started. This is because the platforms were much more fragmented, and there was little standardisation (and thus little compatibility) among them. Furthermore, hardware was being developed at a rapid pace. All these factors made it difficult to re-use any abstractions (such as those provided by a game engine) in the long term. Hardware was also quite limited, so it was necessary to build a game's components from scratch and custom-tailor them to the target platform to ensure the game ran optimally.

The earliest software that may constitute a game engine was in-house for use with first-party software. These engines were far more limited in scope than modern game engines (what may now be considered \textit{middleware} used by a game engine). Engines for third-parties became prevalent in the 1990s along with the growing popularity and feasibility of 3D graphics. This led to influential engines such as id Tech and Unreal Engine, which are in some form or another still in use with modern games.

Such engines were developed for first-party games, but made available for licensing to third-party developers. Companies shifted their strategy to developing the game and its engine separately. This decoupling enabled the engine to be usable by third parties, and allowed developers to become more specialised.

\end{document}

\documentclass[a4paper, 12pt]{scrartcl}
\usepackage[assign-num=3]{report}

\title{Comparison of Game Engines}

\begin{document}

\maketitle

\section{Introduction}
A game engine is a tool for developers of video games. It is software framework which abstracts away core parts of a video game such as rendering, physics, input handling, sound, AI, etc. It typically includes software libraries and other software in the form of a software development kit (SDK), which contains additional tools revolving around the engine's ecosystem (level editor, asset packager, and so on).

A game engine enables developers to create games more efficiently, since a lot of the fundamental components of a game are available for use out of the box. This saves developers the time of having to implement all these core features from scratch, which could otherwise add a tremendous amount of complexity to the project. For example, game engines may offer cross-platform capabilities in a more or less transparent way to the developers. This removes the necessity to deal with the plethora of different platform-specific APIs, which allows developers and publishers to reach a broader market more easily.

\end{document}

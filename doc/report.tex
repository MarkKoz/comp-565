% !TEX output_directory = ./.temp/report
% !TEX options = --shell-escape

\documentclass[a4paper, 12pt]{scrartcl}
\usepackage{lmodern}
\usepackage[T1]{fontenc}
\usepackage[utf8]{inputenc}

\usepackage{hyperref}
\hypersetup{bookmarksnumbered}
\usepackage{bookmark}

% Remove space reserved for unused fields.
% https://tex.stackexchange.com/a/134857
\makeatletter
\renewcommand{\@maketitle}{\null\vskip 2em
\begin{center}
  \ifx\@subject\@empty \else
    {\subject@font \@subject \par}
    \vskip 1.5em
  \fi
  \titlefont\huge \@title\par
  \vskip .5em
  {\ifx\@subtitle\@empty\else\usekomafont{subtitle}\@subtitle\par\fi}%
\end{center}
\vskip 2em}
\makeatother

\title{Unity Mini Project}
\subject{COMP 565 - Advanced Computer Graphics}
\subtitle{Assignment 2}
\date{}
\author{}

\begin{document}

\maketitle

\section{Introduction}

\section{Approach}
\subsection{User Interface}
The user interface needs to allow the user to select the new primitive's texture and type. There are three choices for each. The UI is divided into two distinct parts: a panel along the left edge for the type buttons, and a panel along the bottom edge for the texture selection.

The two panels are children of the same canvas, which itself is a child of a \textit{UI} object. This object is for organisational purposes and for housing the script component for handling UI interactions.

Each panel has 3 buttons, which display the appropriate provided texture. The type of these textures has to be changed to \textit{Sprite} in order to use them in the UI. The panels have a vertical/horizontal layout group component to evenly space the buttons within the panels. Anchor presets are used to align the panels to the middle left or centre bottom.

Each button has its \texttt{OnClick} handled by the script component in the \textit{UI} object mentioned earlier. The texture buttons use \texttt{UIManager.OnTextureClick} and the primitive type buttons use \texttt{UIManager.OnPrimitiveClick}. From the inspector, an argument is passed to these functions, which is an integer that corresponds to the \texttt{Enum}s \texttt{UIManager.Texture} and \texttt{PrimitiveType}, respectively.

Upon a button being clicked, the appropriate handler is called. It converts the integer argument to the corresponding enum and stores the enum in the public field \texttt{UIManager.texture} or \texttt{UIManager.primitive}.

\subsection{Primitive Creation}

\subsection{Placement Guide}

\subsection{Primitive Destruction}

\section{Challenges}

\section{Lessons Learned}

\end{document}
